\documentclass[12pt]{article}

\usepackage{amsmath}    % need for subequations
\usepackage{graphicx}   % need for figures
\usepackage{verbatim}   % useful for program listings
\usepackage{color}      % use if color is used in text
\usepackage{subfigure}  % use for side-by-side figures
\usepackage{hyperref}   % use for hypertext links, including those to external documents and URLs
\usepackage{amsfonts}
\usepackage{amssymb}
\usepackage{mathtools}
\DeclarePairedDelimiter{\ceil}{\lceil}{\rceil}

\usepackage{blindtext}
\usepackage{enumitem}
\usepackage{xcolor}
\usepackage{listings}
\lstset{numbers=left,
emph={%
   {:-}%
     },emphstyle={\color{red}\bfseries\underbar}%
}%

% don't need the following. simply use defaults
\setlength{\baselineskip}{16.0pt}    % 16 pt usual spacing between lines

\setlength{\parskip}{3pt plus 2pt}
\setlength{\parindent}{20pt}
\setlength{\oddsidemargin}{0.5cm}
\setlength{\evensidemargin}{0.5cm}
\setlength{\marginparsep}{0.75cm}
\setlength{\marginparwidth}{2.5cm}
\setlength{\marginparpush}{1.0cm}
\setlength{\textwidth}{150mm}

\begin{comment}
\pagestyle{empty} % use if page numbers not wanted
\end{comment}

% above is the preamble

\begin{document}

\begin{center}
{\large COMP3411 : Assignment 2} \\ % \\ = new line
Kevin Luo z5061845 \\
\end{center}

\section{Question 1: Search algorithms for 15-puzzle}

\subsection{}

\begin{table}[h]
\centering
\caption{States expanded}
\label{my-label}
\begin{tabular}{llllll}
     & start10 & start12 & start20 & start30 & start40 \\
UCS  & 2565    & Mem     & Mem     & Mem     & Mem     \\
IDS  & 2407    & 13812   & 5297410 & Time    & Time    \\
A*   & 33      & 26      & 915     & Mem     & Mem     \\
IDA* & 29      & 21      & 952     & 17297   & 186115
\end{tabular}
\end{table}

\subsection{}

\begin{description}[font=$\bullet$\scshape\bfseries]
  \item [UCS] UCS is the worst algorithm. It exceeded its memory at start12. \textbf{Slow}
  \item [IDS] This performed slightly better than UCS but its memory also grew exponetially. However it never ran out of memory, instead its time expired. \textbf{Slow}
  \item [A*] A* has a very similar time to IDA* up to start20, afterwards, its memory usage blows up and is unable to calculate larger values. \textbf{Medium Speed}
  \item [IDA*] It is clear that IDA* is the most efficient in terms of time and memory usage, due to it being the only one to be able to calculate past start30. \textbf{Medium Speed}
\end{description}


\newpage

\section{Heuristic Path Search}

\subsection{}

\begin{table}[h]
\centering
\caption{Search Path}
\label{my-label}
\begin{tabular}{|l|l|l|l|l|l|l|}
\hline
                                                        & \multicolumn{2}{l|}{start50}                                                                                            & \multicolumn{2}{l|}{start60}                                                                                          & \multicolumn{2}{l|}{start64}                                                                                              \\ \hline
IDA*                                                    & 50                                                    & 14642512                                                        & 60                                                    & 321252368                                                     & 64                                                    & 1209086782                                                        \\ \hline
\begin{tabular}[c]{@{}l@{}}1.2\\ 1.4\\ 1.6\end{tabular} & \begin{tabular}[c]{@{}l@{}}52\\ 66\\ 100\end{tabular} & \begin{tabular}[c]{@{}l@{}}191438\\ 116174\\ 34647\end{tabular} & \begin{tabular}[c]{@{}l@{}}62\\ 82\\ 148\end{tabular} & \begin{tabular}[c]{@{}l@{}}230861\\ 3673\\ 55626\end{tabular} & \begin{tabular}[c]{@{}l@{}}66\\ 94\\ 162\end{tabular} & \begin{tabular}[c]{@{}l@{}}431033\\ 188917\\ 2358520\end{tabular} \\ \hline
Greedy                                                  & 164                                                   & 5447                                                            & 166                                                   & 1617                                                          & 184                                                   & 2174                                                              \\ \hline
\end{tabular}
\end{table}

\subsection{}

\lstinputlisting{2b.pl}\\
Line removed is line (10) and the lines added are (11-12). These introduce a new var W, and is used to modify F1 depending on the equation given.\\

\subsection{}
Refer to table in 2.1

\subsection{}
\begin{description}[font=$\bullet$\scshape\bfseries]
  \item [IDA*] Slowest algorithm by far. It only adds the cost of the heuristic. Has quickest path.
  \item [1.2] We implemented the Heuristic Path Search with an Iterative Deepening solution. It is able to generate the graph fairly quickly while its solution is just below optimal.
  \item [1.4] Same as above, runs faster than 1.2 and has fewer nodes expanded but its solution is no longer optimal.
  \item [1.6] Same as 1.2, with even fewer nodes expanded but has very bad optimal path.
  \item [Greedy] Fastest solution while generating the least amount of nodes but it resulted in the longest path solution.
  \item [Outlier] start60 produced some weird solutions where the nodes expanded were less than their start 50 cousins.
\end{description}



\section{Maze Search Heuristics}
\subsection{}
Manhattan heuristic
\begin{gather*}
  h(x,y,x_G,y_G)= |x-x_G| + |y-y_G|
\end{gather*}

\subsection{}
\subsubsection{}
No, a heuristic must always underestimate or be equal to the cost of a goal. Below is a counter example where the quickest route with diagonal movement is 2 moves but the straight line heuristic gives:
\begin{align*}
  \sqrt{2^2 + 1^2} = \sqrt{5} = 2.23
\end{align*}
\begin{table}[h]
\centering
\caption{Counter example}
\label{my-label}
\begin{tabular}{|l|l|l|}
\hline
  &  & G \\ \hline
S &  &   \\ \hline
\end{tabular}
\end{table}

\subsubsection{}
No. The Manhattan heuristic is no longer admissable in the case of diagonal movement due to the pythagoras theorem - the root of the two squared sides is longer than the hypothenus.
\subsubsection{}
Chebyshev distance
\begin{gather*}
  h(x,y,x_G,y_G)= max(|x-x_G|, |y-y_G|)
\end{gather*}

\newpage

\section{Graph Paper Grand Prix}

\subsection{}
\begin{table}[h]
\centering
\caption{My caption}
\label{my-label}
\begin{tabular}{|l|l|l|}
\hline
n  & Sequence & Number of actions \\ \hline
1  &  +-      & 2                 \\ \hline
2  &  +o-    & 3                  \\ \hline
3  &  +oo-    & 4                   \\ \hline
4  &  ++- -    & 4                  \\ \hline
5  &  ++-o-   & 5                  \\ \hline
6  &  ++o- -   & 5                  \\ \hline
7  &  ++o-o-  & 6         \\ \hline
8  &  ++oo- -  & 6                  \\ \hline
9  &  +++- - -  & 6                  \\ \hline
10 &  +++- -o- & 7                  \\ \hline
11 &  +++-o- - & 7                  \\ \hline
12 &  +++o- - -  & 7                  \\ \hline
13 &  +++o- -o-  & 8                  \\ \hline
14 &  +++o-o- -  & 8                  \\ \hline
15 &  +++oo- - -  & 8                  \\ \hline
16 &  ++++- - - -  & 8                  \\ \hline
17 &  ++++- - -o-         & 9                  \\ \hline
18 &  ++++- -o- -        & 9                  \\ \hline
19 &  ++++-o- - -         & 9                  \\ \hline
20 &  ++++o- - - -         & 9                  \\ \hline
21 &  ++++o- - -o-        & 10                  \\ \hline
\end{tabular}
\end{table}
\subsection{}
Let Number of Actions (NoA) from the table above be the length of the sequence. NoA follows the given identity where $s$ is the maximum speed - number of $+$. This holds true for $s(s+1)$ no matter how many $o$'s (rests) are present. We then have the identity $s^2$ for the rest of the sequences, ie when n is a perfect square.

\subsection{}

We start with some velocity $k\geq 0$ and a destination $n$ such that $n \geq \frac{k(k-1)}{2}$. We then have that the deceleration time for $M(n,0)$ would be the total distance up to some $x$, minus the time it too to accelerate to k:

\begin{align*}
  M(x,0) = \lceil 2\sqrt{x} \rceil - k
\end{align*}

We have that x is the total distance to accelerate to k.

\begin{align*}
  \frac{k(k+1)}{2} + n
\end{align*}

Substituting into the original at $x,k$, we have:
\begin{align*}
  M(n, k) = \ceil[\bigg]{2\sqrt{n + \frac{k(k+1)}{2}}} - k\\
\end{align*}

\subsection{}
Similar to 4.3, we have some velocity $k\geq 0$ but a destination $n$ such that $n < \frac{k(k-1)}{2}$. Our destination has extended over the goal and we have to reverse back towards G. We have $x>n$ such that \\$M(n,k) =$ time extended past goal + time to reverse - time to accelerate (k).\\ This gives us:

\begin{align*}
  M(n, k) = & \ceil[\bigg]{2\sqrt{\frac{k(k+1)}{2} + \frac{k(k-1)}{2}}} + \ceil[\bigg]{2\sqrt{\frac{k(k-1)}{2} - n}} - k\\
  M(n, k) = & \ceil[\bigg]{2\sqrt{\frac{k(k-1)}{2}}} + k
\end{align*}

\subsection{}
Similarly to the Chebyshev distance but without the absolute values since we can have negative acceleration
, we have:
\begin{align*}
  h(r,c,u,v,r_G,c_G) = \mbox{max}(M(r_G - r, u),M(c_G-c, v))
\end{align*}

where M is the equation from 4.3


\end{document}
